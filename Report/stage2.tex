\documentclass[12pt]{article}
\usepackage[margin=1in]{geometry}
\usepackage[all]{xy}
\usepackage{setspace}
\usepackage{enumitem}
%\usepackage{biblatex}
\author{
  Souleyman Boudouh, Clément Charmillot, Emilien Guandalino\\
}



\usepackage{amsmath,amsthm,amssymb,color,latexsym}
\usepackage{geometry}        
\geometry{letterpaper}    
\usepackage{graphicx}


\begin{document}

\title{CS-471 Course Project Proposal}
\date{}
\maketitle

\vspace{-3em}



\section{Research Problem}
• Updates or changes to the problem and the reasons.

• New recent work addressing the same problem.

• Newly discovered related and intriguing problems.

\section{Questions, Hypotheses and Test}

• Current status of all questions and hypotheses, such as tested, planned, stuck, not related, or abandoned.

• New questions, for instance, anomalies encountered during experiments or literature review.

• New hypotheses

• Methodology details, including machine configuration, software usage, and benchmarks to be conducted.

\section{Progress, Result and Analysis}

Results may vary with the type of research question. Below are suggestions for a given type of research question:

• Metric Modeling: Include your model, analysis steps, experimental validation results, and error analysis.

• Solution Proposal: Test results supporting your design decisions.

• Result Reproduction: Experimental findings and comparisons with previous studies.

• Workload Characterization: Measurements of metrics.

• Methodology Examination: Comparative measurements using different methodologies.

\section{Challenges}
• Difficulties encountered, whether in experimentation, data gathering, or literature review.

• Assistance requests for your project.
\section{Plan}

• Updates on team composition (e.g., team member changes).

• Reprioritization of questions, hypotheses, and tests.

• Schedule adjustments (compared to the initial proposal).

• Updates on the final goal (compared to the initial proposal).


\bibliographystyle{IEEEtran}
\bibliography{ref}

\end{document}
